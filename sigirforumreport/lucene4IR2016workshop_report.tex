\documentclass[12pt]{article}
\usepackage{fancyhdr}
\usepackage{url}
\usepackage{color}
\usepackage{mathtools}

 \topmargin -0.8cm
 \oddsidemargin -0.7cm

 \textwidth 17.5cm
 \textheight 22.6 cm


\pagestyle{fancy} {\fancyhead{}
\fancyfoot[c]{\small{\rule{17.5cm}{1pt}}}}

\begin{document}
\title{\vspace{-2.5cm}
\begin{center}
\textbf{\small{Lucene4IR Workshop Report}}\\\vspace{-0.5cm} \rule{17.5cm}{1pt}
\end{center}
\vspace{1cm}\textbf{Report on the Lucene4IR workshop: Developing Information Retrieval Evaluation Resources using Lucene (L4IR2016) }}

\newcommand{\todo}[1]{\textcolor{red}{#1}}
\author{
Leif Azzopardi$^{1}$, Yashar Mosfeghi$^{2}$, Martin Halvey$^{1}$, \\
Krisztian Balog$^{3}$, Emanuele Di Buccio $^4$, Juan Manual Fernandez Luna $^{5}$,\\
 Charlie Hull$^{6}$, Jake Mannix$^{7}$, Sauparna Palchowdhury$^{8}$\\
    $^{1}$ {\small University of Strathclyde  \emph{ \{Leif.Azzopardi,Martin.Halvey\}@strath.ac.uk}}\\
    $^{2}$ {\small University of Glasgow\emph{\small Yashar.Mosfeghi@glasgow.ac.uk}}\\
	$^{3}$ {\small University of Stavanger \emph{\small krisztian.balog@uis.no}}\\
	$^{4}$ {\small University of Padova \emph{\small dibuccio@dei.unipd.it}}\\
	$^{5}$ {\small University of Granda \emph{\small jmfluna@decsai.ugr.es}}\\
	$^{6}$ {\small Flax \emph{\small charlie@flax.co.uk}}\\
	$^{7}$ {\small LucidWorks \emph{\small jake.mannix@lucidworks.com}}\\
	$^{8}$ {\small NIST \emph{\small sauparna.palchowdhury@nist.gov}}
}

\begin{sloppypar}

\maketitle \thispagestyle{fancy} 
\abstract{
The workshop and hackathon on developing Information Retrieval Evaluation Resources using Lucene (L4IR) was held on the 8th and 9th of September, 2016 at the University of Strathclyde in Glasgow, UK and funded by the ESF Elias Network. The event featured three main elements: (i) a series of keynote and invited talks on Lucene in action in industry, in teaching and learning environments, and evaluation forums. (ii) planning, coding and hacking where a number of groups created modules and infrastructure to use Lucene to undertake TREC based evaluations. And (iii) a number of breakout groups discussing challenges, opportunities and problems in bridging the divide between academia and industry, and how we can use Lucene and the resources created in teaching and learning IR evaluation. The event was composed of a mix and blend of academics, experts and students wanting to learn, share and create evaluation resources for the community. The hacking was intense and the discussions lively creating the basis of many useful tools and raising numerous issues. However, by adopting and contributing to most widely used and supported Open Source IR toolkit, it was clear that there were many benefits for academics, students, researchers, developers and practitioners - providing a basis for stronger evaluation practices, increased reproducibility, more efficient knowledge transfer, greater collaboration between academia and industry, and shared teaching and training resources.}

\maketitle



\section{Introduction}
Lucene and its expansions, Solr and ElasticSearch, represent the major open source Information Retrieval toolkits used in Industry. However, there is a lack of coherent and coordinated documentation that explains from an experimentalist's point of view how to use Lucene to undertake and perform Information Retrieval Research and Evaluation. In particularly, how to undertake and perform TREC based evaluations using Lucene. Consequently, the objective of this event was to bring together researchers and developers to create a set of evaluation resources showing how to use Lucene to perform typical IR operations (i.e.  indexing, retrieval, evaluation, analysis, etc.) as well as how to extend, modify and work with Lucene to extract typical statistics, implement typical retrieval models. Over the course of the workshop participants shared their knowledge with each other creating a number of resources and guides along with a road map for future development.


%!TEX root = lucene4IR2016workshop_report.tex
\section{Keynotes and Invited Talks}
During the course of the workshops a series of talks on how Lucene is being used in Industry, Teaching and for Evaluation along with more technical talks on the inner workings of how Lucene's scoring algorithm works and how learning to rank is being included into Solr\footnote{\scriptsize{Slides are available from \url{www.github.com/leifos/lucene4ir}}}. 

\subsection*{Introduction Talk: Why are we here?}
{\bf Leif Azzopardi, University of Strathclyde}:
Leif explained how after attending the lively Reproducibility workshop at ACM SIGIR 2015, he wondered where the Lucene team was, and why, if Lucene and the community is so big, why they don't come to IR conferences - he posited that perhaps we haven't been inclusive and welcoming to such a large community of search practitioners as we could, and have perhaps failed to transfer our knowledge into one of the largest open source toolkits available. He argued that if we as academics want to increase our impact then we need to improve how we transfer our knowledge to industry. One way is working with large search engines, but what about other industries and organisations that need search and use Lucene based tools? He argued that we need to start speaking the same language i.e. work with Lucene et al and look for opportunities on how we can contribute and develop resources for training and teaching IR and how to undertake evaluations and data science using widely used, supported and commonly accepted Open Source toolkits like Lucene. He described how this workshop was a good starting point and opportunity to explore how academica and industry can better work together, where we can identify common goals, needs and resources that are needed. 

%!TEX root = Lucene4IR2016workshop_report.tex
\subsection*{Keynote Talk: Apache Lucene in Industry} 
{\bf Charlie Hull, Flax}: In his talk, Charlie first introduced Flax, and how it evolved over the years. Charlie explained that they have been building search applications using open search software since 2001. Their focus is on building, tuning and supporting fast, accurate and highly scalable search, analytics and Big Data applications. They are partners with Lucidworks, leading Lucene specialists and committers. When Lucene first came out clients were at reluctant to adopt open source, but nowadays it has been much more acceptable. Charlie notes that now you don't have to explain to clients what open source software is, and why it should be used. He described how Lucene-based search engines have risen in use - and that search and data analytics are available to those without six figure budgets. Charlie points out that Lucene is appealing because it is the most widely used open source search engine, which is hugely flexible, feature rich, scalable and performant. It is supported by a large and healthy community and backed by the Apache Software Foundation. Many of world's largest companies use Lucene including Sony, Siemens, Tesco, Cisco, Linkedin, Wikipedia, WordPress and Hortonworks. Charlie notes that they typically don't use Lucene directly, instead they use the search servers, built on top of Lucene, i.e. Apache Solr (which is mature, stable, and crucially highly scalable), or ElasticSearch (easy to get started with, great analytics, scalable). He contrasts these products with some of the existing toolkits in IR, and remarks on the latter, that ``no one in industry has ever heard of them!''. So even though they have the latest research encoded within them, it is not really viable for businesses to adopt them, especially as support for such toolkits is highly limited. He recommends that IR research needs to be within Lucene-based search services for it to be used and adopted. 

%Charlie then described a number of projects that they have been working on. (1) and (2)..

Based on Charlie's experience he provided us with a number of home truths:
\begin{itemize}
	\item Open source does not mean cheap 
	\item Most Search engines are the same (in terms of underlying features and capabilities)
	\item Complex features are seldom used - and often confusing
	\item Search testing is rarely comprehensive
	\item Good search developers are hard to find
\end{itemize}


Charlie reflected on these points considering how we can do better. First, learn what works in industry and how industry are using search - there are lots of research challenges which they rarely get to solve and address but solutions to such problems would have real practical value. Second, improve Lucene et al with ideas from academia - faster - for example, it took years before BM25 replaced TFIDF as the standard ranking algorithm, where as toolkits like Terrier already have infrastructure for Learning to Rank, while this is only just being developed in Lucene. Third, he pointed out that testing and evaluation of Lucene based search engines is very limited, and that thorough evaluations by search developers is poor. He argued that this could be greatly improved, if academics and researchers, contributed to the development of evaluation infrastructure, and transferred their knowledge to practitioners on how to evaluate. Lastly, he pointed that the lack of skilled and knowledgable search developers was problematic - having experience with Lucene, Solr and ElasticSearch are highly marketable skills, especially, when there is a growing need to process larger and larger volumes of data - big data requires data scientists! So there is the pressing need to create educational resources and training material for both students and developers. 


%!TEX root = lucene4IR2016workshop_report.tex
\subsection*{Using Lucene for Teaching and Learning IR: The 
University of Granada case of study ?} 
{\bf  Prof. Juan Manual Fernandez Luna (University of Granda) }:
In this talk, we shall describe how the University of Granada is 
supporting teaching and learning Information Retrieval (TLIR) discipline 
across different courses in the Computer Science studies (degree and 
masters), and how this is done by means of the Lucene API. Later we 
shall present our thoughts about how Lucene could be used inTLIR context 
and present some proposals for improving the Lucene experience, both for 
students and lecturers.

%!TEX root = lucene4IR2016workshop_report.tex
\subsection*{Black Boxes are Harmful}
{\bf Sauparna Palchowdhury (NIST)}:

Having seen students and practitioners in the IR community grapple
with abstruse documentation accompanying search systems and their use
as a black box, Sauparna, in his talk, argued why Lucene is a useful
alternative and how and why we must ensure it does not become another
black box. In establishing his views, he described the pitfalls in an
IR experiment and the ways of mitigation. The suggestions he puts
forth, as a set of best practices, highlights the importance of
evaluation in IR to render an experiment reproducible and repeatable
and the need for a well-documented system with correct implementations
of search algorithms traceable to a source in IR literature. In the
absence of such constraints on experimentation students are misled and
learn little from the results of their experiments and it becomes hard
to reproduce the experiments. As an example, the talk cited a wrong
implementation of the \emph{Okapi BM25} term-weighting equation in a
popular research retrieval system (Table \ref{tab:tfxidf}). Following
this was a brief how-to on implementing \emph{BM25} (or any TFxIDF
weighting scheme) in Lucene (Table \ref{tab:lucene}). This also
explained Lucene's way of computing the similarity between two text
documents (usually referred to as `Lucene's scoring') that may be of
use to the student and practitioner interested in using Lucene for IR
experiments.

Some of the points of failures mentioned in the talk are misplaced
test-collection pieces (document-query-qrel triplet), counterintuitive
configuration interfaces of systems, poor documentation that makes
systems look enigmatic and lead to the creation of heuristics passed
around by word-of-mouth, naming confusion (a myriad of TFxIDF model
names), blatant bugs and not knowing how the parser works. As
mitigation, Sauparna listed some of the things he did as an
experimenter. He wrote a script (TRECBOX) to abstract the IR
experiment pipeline and map them to configuration end-points of the
three systems; Terrier, Lucene and Indri. This would enable
documenting an experiment's design in plain text files, that could be
shared and the experiment repeated. He constructed a survey of TFxIDF
variants titled \emph{TFxIDF Repository} ~\cite{rup:TFXIDFRepository}
meant to be a single point of reference to help disambiguate the
variants in the wild. All mentions of term-weighting equations in this
repository are traceable to a source in IR literature. He also shows
how to visually juxtapose evaluation results obtained using a
permutation of a set of systems, retrieval models and test-collections
on a chart that would act as a sanity check for the system's
integrity. As a part of these investigations he modified Lucene for
use with TREC collections (the mod was named LTR) which is available
for others to use. The ``mod'' is also accompanied by notes to augment
Lucene's documentation. The gamut of Sauparna's work is to be found
online ~\cite{rup:IR}.

Lucene's documentation does not use well-defined notation to represent
its way of computing the similarity score between a pair of
documents. The notation below is Lucene's scoring equation, lifeted
from Lucene's documentation. It uses actual function names that are to
be found in Lucene's source code;

$score(Q,D) = coord(Q,D) \cdot qnorm(Q) \cdot \displaystyle\sum_{T \in Q} (tf(T \in D) \cdot idf(T)^2 \cdot boost() \cdot norm(T, D))$\\

Sauparna's explanation begins with a well-defined, generalized,
notation for Lucene's scoring in step with the definition from
Lucene's documentation ~\cite{Lucene:6.2.1:Scoring};

$score(Q,D) = f_{c}(Q,D) \cdot f_{q}(Q) \cdot \displaystyle\sum_{T \in Q \cap D}(tf(T_{k}) \cdot df(T_{k}) \cdot f_{b}(T_{k}) \cdot f_{n}(T_{k},D)))$\\

He picks two popular TFxIDF variants, breaks them down into meaningful
components (a term-frequency transformation, a transformation on the
document-frequency and a length normalization coefficient) and plugs
these components into Lucene's equation. The components in Lucene's
equation that are left unused are replaced by the integer $1$,
meaning, the functions return $1$; which has no effect on the chain of
multiplications. Table \ref{tab:tfxidf} lists the variants and
components and Table \ref{tab:lucene} shows where the components were
transplanted to.

\begin{table}
  \centering
  \small
  \begin{minipage}[t]{0.65\textwidth}
    
    \begin{tabular}{lcc}
      \multicolumn{3}{c}{TFxIDF Variants: what's correct and what's not.}\\
      \hline\hline
      \\
      Name & $w_{ik}$ & $w_{jk}$\\
      \hline
      \\
      BM25(A)
      & $\frac{f_{ik}}{k_{1}((1-b)+b\frac{dl_{i}}{avdl})+f_{ik}} \times \log(\frac{N-n_{k}+0.5}{n_{k}+0.5})$
      & $\frac{(k_{3}+1)f_{jk}}{k_{3}+f_{jk}}$ \\
      \\
      BM25(B)
      & $\frac{(k_{1}+1)f_{ik}}{k_{1}((1-b)+b\frac{dl_{i}}{avdl})+2f_{ik}} \times \log(\frac{N-n_{k}+0.5}{n_{k}+0.5})$
      & $\frac{(k_{3}+1)f_{jk}}{k_{3}+f_{jk}}$ \\
      \\\hline
      \\
      Okapi BM25
      & $\frac{(k_{1}+1)f_{ik}}{k_{1}((1-b)+b\frac{dl_{i}}{avdl})+f_{ik}} \times \log(\frac{N-n_{k}+0.5}{n_{k}+0.5})$
      & $\frac{(k_{3}+1)f_{jk}}{k_{3}+f_{jk}}$ \\
      \\
      components & $T \times I$ & $Q$ \\
      \\\hline
      \\
      SMART dtb.nnn
      & $\frac{(1+\log(1+\log(f_{ik}))) \times \log(\frac{N+1}{n_{k}})}{1-s+s \cdot \frac{b_{i}}{avgb}}$
      & $f_{jk}$ \\
      \\
      components & $T \times I \div L$ & $Q$ \\
      \\\hline\hline
      \label{tab:tfxidf}
    \end{tabular}
    \caption{ The similarity score;
      $score(D_{i},D_{j})=\sum_{k=1}^{t}(w_{ik} \cdot w_{jk})$
      $\forall i \neq j$, combines the weight of a term $k$ over the
      $t$ terms which occur in document $D_{i}$ and $D_{j}$. Since a
      query can also be thought of as a document in the same vector
      space, the symbol $D_{j}$ has been used to denote a query
      without introducing another symbol, say, $Q$. BM25(A) and
      BM25(B) are the two incorrect implementations found in a popular
      retrieval system. Comparing them to \emph{Okapi BM25} on the
      third row shows that A has the $k_{1}+1$ factor missing in the
      numerator, and B uses twice the term-frequency $2f_{ik}$ in the
      denominator. Neither can they be traced to any source in IR
      literature, nor does the system's documentation say anything
      about them. The \emph{Okapi BM25} and the \emph{SMART dtb.nnn}
      variants are known to be effective formulations developed by
      trial and error over eight years of experimentation at TREC 1
      through 8. Their forms have been abstracted using capital
      letters to show how these components fit in Lucene's term-weight
      expression.}

  \end{minipage}
\end{table}

\begin{table}[bht!]
  \centering
  \small
  \begin{minipage}[t]{0.94\textwidth}

    \begin{tabular}{lccccccccccccc}
      \multicolumn{14}{c}{IMPLEMENTING TFxIDF VARIANTS IN LUCENE}
      \\
      \hline\hline

      Lucene    & $f_{c}(Q,D)$ & $\cdot$  & $f_{q}(Q)$
      & $\cdot$ & $\displaystyle\sum_{T \in Q \cap D}($  & $tf(T_{k})$
      & $\cdot$ & $df(T_{k})$  & $\cdot$  & $f_{b}(T_{k})$
      & $\cdot$ & $f_{n}(T_{k}, D_{j})$   & $)$ \\
      
      BM25      & $1$          &  $\cdot$ & $1$
      & $\cdot$ & $\displaystyle\sum_{T \in Q \cap D}($  & $T$
      & $\cdot$ & $I$          & $\cdot$  & $Q$
      & $\cdot$ & $1$          & $)$ \\

      dtb.nnn   & $1$          & $\cdot$  & $1$
      & $\cdot$ & $\displaystyle\sum_{T \in Q \cap D}($  & $T$
      & $\cdot$ & $I$          & $\cdot$  & $Q$
      & $\cdot$ & $L$          & $)$ \\

      \hline\hline
    \end{tabular}

    \caption{\small Plugging components of the TFxIDF equation into
      Lucene's scoring equation; the first row is the generalized form
      and the following two rows show the components of two popular
      TFxIDF equations transplanted to Lucene's equation. Table
      \ref{tab:tfxidf} specifies what the capital letters represent.}

    \label{tab:lucene}

  \end{minipage}
\end{table}


Making a reference to the SIGIR 2012 tutorial on \emph{Experimental
  Methods for Information
  Retrieval}~\cite{Metzler:2012:EMI:2348283.2348534}, Sauparna states
that we need to take a more rigorous approach to the IR experimental
methodology. A list of best practices was recommended that would add
more structure to IR experiments and prevent the use of systems as
black boxes. These were:

\begin{itemize}
\item Record test-collection statistics.
\item Provide design documentation for systems.
\item Use a consistent naming scheme and a well-defined notation.
\item Build an evaluation table to be used as a sanity-check.
\item Isolate sharable experimental artefacts.
\item Ensure that implementations are traceable to a source in IR
  literature.
\end{itemize}

In conclusion, Sauparna suggests that if we, the IR research
community, are to build and work with Lucene, then it would be helpful
to consider these points when introducing new features into Lucene.




%!TEX root = lucene4IR2016workshop_report.tex
\subsection*{ Deep Dive into the Lucene Query/Weight/Scorer Java Classes}
{\bf Jake Mannix, Lucidworks}:
In this more technical talk, Jake explained how Lucene scores a query, and what classes are instantiated to support the scoring. Jake described, first, at a high level how to do scoring modification to Lucene-based systems, including some ``Google''-like questions on how to score efficiently. Then, he went into more details about the BooleanQuery class and is cousins, showing where the Lucene API allows for modifications of scoring with pluggable Similarity metrics and even deep inner-loop, where ML-trained ranking models could be instantiated - \emph{if you're willing to do a little work}.


%!TEX root = lucene4IR2016workshop_report.tex
\subsection*{Learning to Rank with Solr} 
{\bf Diego Ceccarelli, Bloomberg}
On day two of the workshop, Diego started his talk by explaining that tuning Lucene/Solr et al is often performed by ``experts'' who hand tune and craft the weightings used for the different retrieval features. However, this approach is manual, expensive to maintain, and based on intuitive, rather than data. His working goal behind this project was to automate the process. He described how this motivated the use of Learning To Rank, a technique that enables the automatic tuning of a information retrieval system by applying machine learning when estimating parameters. He points out that sophisticated models can make more nuanced ranking decisions than a traditional ranking function when tuned in such a manner. During his talk, Diego presented the key concepts of Learning to Rank, how to evaluate the quality of the search in a production service, and then how the Solr plugin works. At Bloomberg, they have integrated a learning to rank component directly into Solr (and released the code as Open Source), enabling others to easily build their own Learning To Rank systems and access the rich matching features readily available in Solr. 





\section{Discussion}
During the course of the workshop, two breakout groups were formed to discuss how we can use Lucene when teaching and learning, and what were the main challenges in bridging the industry/academia along with what opportunities it could bring about. Finally, we asked participants to provide some feedback on the event.



\subsection{Teaching and Learning}



\todo{Juanma}

\todo{Krisztian}


\todo{Martin}


How do you go about teaching IR? What level?

What kinds of things do you need/want from such resources?


How do we see Lucene fitting in? Benefits to students?


\begin{table}[]
\centering
\caption{Attempt at encoding the picture}
\label{my-label}
\begin{tabular}{|l|l|l|}
\hline
Apps                    & High Level                                                                                                            & Low Level                                                                                                                                                           \\ \hline
IndexerApp              & \begin{tabular}[c]{@{}l@{}}Modify how the indexer is performed\\ i.e. different tokenizers, parsers, etc\end{tabular} & Can modify parsers, tokenizers, etc                                                                                                                                 \\ \hline
IndexAnalyzerApp        & Inspect the influence of indexer                                                                                      &                                                                                                                                                                     \\ \hline
RetrievalApp            & \begin{tabular}[c]{@{}l@{}}Try out different retrieval algorithms\\ Change retrieval parameters\end{tabular}          & Implement new retrieval algorithms                                                                                                                                  \\ \hline
trec\_eval              & Measure the performance                                                                                               &                                                                                                                                                                     \\ \hline
ResultAnalyzerApp       & Inspect and analyze the results returned                                                                              & \begin{tabular}[c]{@{}l@{}}Customise the analysis, put out other \\ statistics of interest\end{tabular}                                                             \\ \hline
ExampleApp              &                                                                                                                       & \begin{tabular}[c]{@{}l@{}}Examples of how to work with the Lucene\\ index, to make modifications\end{tabular}                                                      \\ \hline
Batch Retrieval Scripts & \begin{tabular}[c]{@{}l@{}}Configure to run a series of standard\\ batch experiments\end{tabular}                     & \begin{tabular}[c]{@{}l@{}}Customize to run specific retrieval\\ experiments\end{tabular}                                                                           \\ \hline
RetrievalShellApp       & n/a                                                                                                                   & \begin{tabular}[c]{@{}l@{}}Customize to implement retrieval algorithms \\ outwith the Lucene scorer i.e. a simple scorer \\ assuming term independence\end{tabular} \\ \hline
\end{tabular}
\end{table}
\subsection{Challenges and Opportunities}


\todo{Some challenges from charlie}
- lack of good real-world test data for researchers. Companies need to provide this
- many open source search companies are small and therefore find it hard to support apprenticeships, internships etc. or access funding such as KTPs
- Lucene community can be hard to enter - learning curve can be steep, this is a very large and complex project

%!TEX root = lucene4IR2016workshop_report.tex
\subsection{Feedback}


%!TEX root = lucene4IR2016workshop_report.tex
\section{Resources}
As part of the workshop numerous attendees contributed to the Lucene4IR GitHub Repository - \url{http://github.com/leifos/lucene4ir/}. In the repository, three main applications were developed and worked on:
\begin{itemize}
	\item IndexerApp - enables the indexing of several different TREC collections, e.g. TREC123 News Collections, Aquaint Collection, etc.
	\item RetrievalApp - a batch retrieval application when numerous retrieval algorithms can be configured, e.g. BM25, PL2, etc
	\item ExampleStatsApp - an application that shows how you can access various statistics about terms, documents and the collection. e.g. how to access the term posting list, how to access term positions in a document, etc.
	\end{itemize}
	
In the repository, a sample test collection (documents, queries and relevance judgements) was provided (CACM), so that participants could try out the different applications.

During the workshop, a number of different teams undertook various projects:
\begin{itemize}
	\item Customisation of the tokenisation, stemming and stopping during the indexing process: this enabled the IndexerApp to be configured so that the collections can be indexed in different ways - the idea being that students would be able to vary the indexing and then see the effect on performance.
	\item Implementation of other retrieval models: inheriting from Lucene's BM25Similiarity Class,  BM25 for Long documents was implemented BM25L\cite{}, OKAPI BM25's was also implemented to facilitate the comparison between how it is currently implemented in Lucene versus an implementation of the original BM25 weighting function~\cite{}.
	\item Rather than scoring through Lucene's mechanics, others attempted to implement BM25 by directly accessing the inverted index - again to provide a comparison in terms of both efficiency and effectiveness for scoring queries.
	\item A QueryExpansionRetrievalApp
	\item Hacking the innerloop
	\item Additional Examples on how to access and work with Lucene's index and 
\end{itemize}


The break-out group focused on inner loop scoring wanted to try something that was simultaneously simple, practical, and yet required some inner loop scoring magic.  Based on the interests of the group members, we decided on "cross-field phrase queries": an extension of the idea of a sloppy phrase query where the "slop" allowed for a pair of terms occurring in *different* fields to be part of a phrase (but with a parametrizably lower score than terms in the same field).  We worked out the design (delegating most of the work to Query / Weight / Scorer classes already in Lucene, but then combining them together across fields), and stepped through much of the iteration implementation.  While we got most of the "plumbing" done, we only had enough time for our "score()" method to be implemented as naively as imaginable, and did not get it fully working in the time of the workshop.  Some participants expressed interest in working on it further, to see how efficient it was, and what effect on scoring it would have (if a QueryParser was configured to explicitly spit out queries of this form sometimes).


\section{Summary}

\section{Acknowledgments}
We thank the European Science Foundation / ELIAS Network for funding the workshop (Grant No. SM 5916). We would also like to thank our speakers as well as Bloomberg, FlaxSearch, LucidWorks and the University of Strathclyde. Finally, we would like to thank all the participants for their contributions to the workshops and hackathon. Also, thanks to Manisha, Guido and Casper for their offline contributions.


\todo{Add references to indri, lemur, terrier, lucene, etc, BM25L}

\bibliography{lucene4IR}{}
\bibliographystyle{acm}

%\bibliography{refWS}
\end{sloppypar}
\end{document}
